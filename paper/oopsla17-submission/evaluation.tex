\section{Evaluation}
\label{sec:evaluation}

\pgfplotstableset{col sep=comma}

\pgfplotstableread{../../data/sp14/op+type+size/linear/results.csv}{\FeatureLinearBench}
\pgfplotstablevertcat{\FeatureLinearBench}{../../data/sp14/op+context+type+size/linear/results.csv}
\pgfplotstablevertcat{\FeatureLinearBench}{../../data/sp14/op+context-has+type+size/linear/results.csv}
\pgfplotstablevertcat{\FeatureLinearBench}{../../data/sp14/op+context-count+type+size/linear/results.csv}
\pgfplotstableread{../../data/sp14/op+type+size/hidden-10/results.csv}{\FeatureHiddenTBench}
\pgfplotstablevertcat{\FeatureHiddenTBench}{../../data/sp14/op+context+type+size/hidden-10/results.csv}
\pgfplotstablevertcat{\FeatureHiddenTBench}{../../data/sp14/op+context-has+type+size/hidden-10/results.csv}
\pgfplotstablevertcat{\FeatureHiddenTBench}{../../data/sp14/op+context-count+type+size/hidden-10/results.csv}
\pgfplotstableread{../../data/sp14/op+type+size/hidden-500/results.csv}{\FeatureHiddenFHBench}
\pgfplotstablevertcat{\FeatureHiddenFHBench}{../../data/sp14/op+context+type+size/hidden-500/results.csv}
\pgfplotstablevertcat{\FeatureHiddenFHBench}{../../data/sp14/op+context-has+type+size/hidden-500/results.csv}
\pgfplotstablevertcat{\FeatureHiddenFHBench}{../../data/sp14/op+context-count+type+size/hidden-500/results.csv}

\pgfplotstableread{../../data/sp14/op+type+size/hidden-10/results.csv}{\HiddenBench}
\pgfplotstablevertcat{\HiddenBench}{../../data/sp14/op+type+size/hidden-25/results.csv}
\pgfplotstablevertcat{\HiddenBench}{../../data/sp14/op+type+size/hidden-50/results.csv}
\pgfplotstablevertcat{\HiddenBench}{../../data/sp14/op+type+size/hidden-100/results.csv}
\pgfplotstablevertcat{\HiddenBench}{../../data/sp14/op+type+size/hidden-250/results.csv}
\pgfplotstablevertcat{\HiddenBench}{../../data/sp14/op+type+size/hidden-500/results.csv}
% \pgfplotstablevertcat{\HiddenBench}{../../data/sp14/op+context-count+type+size/hidden-10/results.csv}
% \pgfplotstablevertcat{\HiddenBench}{../../data/sp14/op+context-count+type+size/hidden-25/results.csv}
% \pgfplotstablevertcat{\HiddenBench}{../../data/sp14/op+context-count+type+size/hidden-50/results.csv}
% \pgfplotstablevertcat{\HiddenBench}{../../data/sp14/op+context-count+type+size/hidden-100/results.csv}
% \pgfplotstablevertcat{\HiddenBench}{../../data/sp14/op+context-count+type+size/hidden-250/results.csv}
% \pgfplotstablevertcat{\HiddenBench}{../../data/sp14/op+context-count+type+size/hidden-500/results.csv}

\pgfplotstableread{../../data/sp14/baseline.csv}{\SpringBench}
\pgfplotstablevertcat{\SpringBench}{../../data/sp14/ocaml/results.csv}
\pgfplotstablevertcat{\SpringBench}{../../data/sp14/mycroft/results.csv}
\pgfplotstablevertcat{\SpringBench}{../../data/sp14/sherrloc/results.csv}
\pgfplotstablevertcat{\SpringBench}{../../data/sp14/op+type+size/linear/results.csv}
\pgfplotstablevertcat{\SpringBench}{../../data/sp14/op+type+size/decision-tree/results.csv}
\pgfplotstablevertcat{\SpringBench}{../../data/sp14/op+type+size/hidden-10/results.csv}
\pgfplotstablevertcat{\SpringBench}{../../data/sp14/op+type+size/hidden-250/results.csv}

\pgfplotstableread{../../data/fa15/baseline.csv}{\FallBench}
\pgfplotstablevertcat{\FallBench}{../../data/fa15/ocaml/results.csv}
\pgfplotstablevertcat{\FallBench}{../../data/fa15/mycroft/results.csv}
\pgfplotstablevertcat{\FallBench}{../../data/fa15/sherrloc/results.csv}
\pgfplotstablevertcat{\FallBench}{../../data/fa15/op+type+size/linear/results.csv}
\pgfplotstablevertcat{\FallBench}{../../data/fa15/op+type+size/decision-tree/results.csv}
\pgfplotstablevertcat{\FallBench}{../../data/fa15/op+type+size/hidden-10/results.csv}
\pgfplotstablevertcat{\FallBench}{../../data/fa15/op+type+size/hidden-250/results.csv}


\subsection{Experimental Setup}
\label{sec:experimental-setup}

\subsection{Type Error Slice}
\label{sec:type-error-slice}
The \InSlice feature should be highly predictive --- a fix must change
at least one expression in the type-error slice.
%
Thus, our first experiment seeks to quantify the impact of \InSlice by
comparing the accuracy of a linear model on three sets of features.
%
\begin{enumerate}
\item A baseline with only local syntactic features.
\item extends (1) with \InSlice.
\item discards samples from (1) where \InSlice is \emph{disabled}.
\end{enumerate}
%
The key difference between (2) and (3) is that a model for (2) must
\emph{learn} that \InSlice is a strong predictor.
%
In contrast, a model for (3) must only learn about the syntactic
features, the decision to discard samples where \InSlice is disabled has
already been made by a human.
%
This has a few additional advantages: it reduces the search space by a
factor of 7 on average, and it guarantees that any prediction made by
the classifier can fix the type error.
%
We expect that (2) will perform better than (1) as it has more
information, and that (3) will perform better than (2) as it does not
have to learn the importance of \InSlice.

We tested our hypothesis with a linear model cross-validated ($k=10$)
over the combined SP14/FA15 dataset. We used a learning rate
$\alpha=0.001$, L2 regularization rate $\lambda=0.001$, and mini-batch
size of 200. We trained for a single epoch on feature sets (1) and
(2), and for 8 epochs on (3), so that the total number of training samples
would be roughly equal for each feature set.

\begin{table}[ht]
  \centering
  \begin{tabular}{lrrrr}
    \toprule
    Feature Set  & Top-1  & Top-2  & Top-3  & Recall \\
    \midrule
    Local Syntax & 23.6\% & 42.6\% & 56.3\% & 19.6\% \\
    With Slice   & 46.4\% & 65.0\% & 75.4\% & 30.4\% \\
    Only Slice   & 54.9\% & 71.4\% & 82.5\% & 57.6\% \\
    \bottomrule
  \end{tabular}
  \caption{
    Impact of type-error slice on accuracy.
    \ES{TODO: expand caption}
    \ES{TODO: load these numbers from CSV}
  }\label{tab:type-error-slice}
\end{table}

\autoref{tab:type-error-slice} shows the results of our experiment.
%
As expected the baseline performs the worst, with a mere 23.6\% Top-1
accuracy.
%
Adding \InSlice improves the results substantially with a 46.4\% Top-1
accuracy, demonstrating the importance of a minimal error slice.
%
However, filtering out expressions that are not part of the slice
\emph{further} improves the results to 54.9\% Top-1 accuracy.
%
Clearly, some decisions are too important to be left to a machine.

Note also the jump in Recall when we filter out expressions that are not
part of the error slice.
%
Reducing the search space not only improves our chances of making a
single correct prediction, it also allows us to make \emph{multiple}
correct predictions per program.

Given the decisive benefits of filtering out expressions that do not
belong to the type-error slice, we will assume going forward that
\emph{all} data sets have been filtered.

\subsection{Contextual Features}
\label{sec:contextual-features}

Next, we will investigate the relative impact of the other three classes
of features discussed in \autoref{sec:features}.
%
For this we consider again a baseline of only local syntactic features,
extended by each combination of
%
(1) expression size,
(2) contextual syntactic features, and
(3) typing features.
%
As before we perform a 10-fold cross-validation with a learning rate and
L2 regularization rate of 0.001 and a mini-batch size of 200, but we
train for a full 10 epochs.

\begin{table}[ht]
  \centering
  \begin{tabular}{lrrrr}
    \toprule
    Feature Set                 & Top-1  & Top-2  & Top-3  & Recall \\
    \midrule
    Local Syntax                & 55.0\% & 71.6\% & 82.6\% & 57.7\% \\
    \midrule
    + Size                      & 55.8\% & 72.8\% & 82.5\% & 57.3\% \\
    + Contextual Syntax         & 59.9\% & 77.7\% & 86.4\% & 63.0\% \\
    + Types                     & 62.2\% & 77.7\% & 85.7\% & 62.2\% \\
    \midrule
    + Size + Contextual Syntax  & 60.2\% & 77.9\% & 86.0\% & 62.5\% \\
    + Size + Types              & 62.0\% & 78.2\% & 85.6\% & 62.3\% \\
    + Contextual Syntax + Types & 63.2\% & 80.3\% & 87.9\% & 65.4\% \\
    \midrule
    + All                       & 62.3\% & 80.0\% & 88.0\% & 65.4\% \\
    \bottomrule
  \end{tabular}
  \caption{
    Impact of contextual features on accuracy.
    \ES{TODO: expand caption}
    \ES{TODO: load these numbers from CSV}
  }\label{tab:type-error-slice}
\end{table}


\definecolor{blue1}{HTML}{DEEBF7}
\definecolor{blue2}{HTML}{9ECAE1}
\definecolor{blue3}{HTML}{3182BD}

% \begin{figure}[ht]
% \centering
% \begin{tikzpicture}
% \begin{axis}[
%   % ybar stacked,
%   width=12cm,
%   height=8cm,
%   title={Impact of Feature Set on Accuracy},
%   ylabel={Accuracy},
%   %ymin=0.2,
%   ymax=1,
%   yticklabel={\pgfmathparse{\tick*100}\pgfmathprintnumber{\pgfmathresult}\,\%},
%   ytick style={draw=none},
%   ymajorgrids = true,
%   symbolic x coords={op+type+size, op+context+type+size, op+context-has+type+size, op+context-count+type+size},
%   % enlarge x limits=0.25,
%   xtick=data,
%   xtick style={draw=none},
%   xticklabels={Type, Context-Is, Context-Has, Context-Count},
%   x tick label style={rotate=45},
%   reverse legend,
%   transpose legend,
%   legend style={legend pos = outer north east, legend columns=4},
% ]
% % \addplot[draw=black, fill=blue1] table[x=tool, y=top-1] {\HiddenBench};
% % \addplot[draw=black, fill=blue2] table[x=tool, y expr=\thisrow{top-2} - \thisrow{top-1}] {\HiddenBench};
% % \addplot[draw=black, fill=blue3] table[x=tool, y expr=\thisrow{top-3} - \thisrow{top-2}] {\HiddenBench};

% \addplot[mark options={fill=blue1, scale=1.5}, mark=square*]
%   table[x=features, y=top-1] {\FeatureHiddenFHBench};
% \addplot[mark options={fill=blue2, scale=1.5}, mark=square*]
%   table[x=features, y=top-2] {\FeatureHiddenFHBench};
% \addplot[mark options={fill=blue3, scale=1.5}, mark=square*]
%   table[x=features, y=top-3] {\FeatureHiddenFHBench};
% \addlegendentry{Top 1}
% \addlegendentry{Top 2}
% \addlegendentry{Top 3}
% \addlegendimage{empty legend}
% \addlegendentry{\hiddenFH}

% \addplot[mark options={fill=blue1, scale=1.5}, mark=*]
%   table[x=features, y=top-1] {\FeatureLinearBench};
% \addplot[mark options={fill=blue2, scale=1.5}, mark=*]
%   table[x=features, y=top-2] {\FeatureLinearBench};
% \addplot[mark options={fill=blue3, scale=1.5}, mark=*]
%   table[x=features, y=top-3] {\FeatureLinearBench};
% \addlegendentry{Top 1}
% \addlegendentry{Top 2}
% \addlegendentry{Top 3}
% \addlegendimage{empty legend}
% \addlegendentry{\linear}

% \end{axis}
% \end{tikzpicture}
% \caption{reuslts!}
% \label{fig:results}
% \end{figure}

% \begin{figure}[ht]
% \centering
% \begin{tikzpicture}
% \begin{axis}[
%   ybar stacked,
%   width=12cm,
%   height=8cm,
%   title={Impact of Hidden Layer Size on Accuracy},
%   ylabel={Accuracy},
%   bar width=20pt,
%   %ymin=0.2,
%   ymax=1,
%   yticklabel={\pgfmathparse{\tick*100}\pgfmathprintnumber{\pgfmathresult}\,\%},
%   ytick style={draw=none},
%   ymajorgrids = true,
%   symbolic x coords={op+type+size/hidden-10, op+type+size/hidden-25, op+type+size/hidden-50,
%                      op+type+size/hidden-100, op+type+size/hidden-250, op+type+size/hidden-500},
%   % enlarge x limits=0.25,
%   xtick=data,
%   xtick style={draw=none},
%   xticklabels={\hiddenT, \hiddenTF, \hiddenF, \hiddenH, \hiddenTHF, \hiddenFH},
%   x tick label style={rotate=45},
%   reverse legend,
%   legend style={legend pos = north west},
% ]
% \addplot[draw=black, fill=blue1] table[x=tool, y=top-1] {\HiddenBench};
% \addplot[draw=black, fill=blue2] table[x=tool, y expr=\thisrow{top-2} - \thisrow{top-1}] {\HiddenBench};
% \addplot[draw=black, fill=blue3] table[x=tool, y expr=\thisrow{top-3} - \thisrow{top-2}] {\HiddenBench};
% % \addplot[draw=black, fill=blue1] table[x=tool, y=top-1] {\HiddenBench};
% % \addplot[draw=black, fill=blue2] table[x=tool, y=top-2] {\HiddenBench};
% % \addplot[draw=black, fill=blue3] table[x=tool, y=top-3] {\HiddenBench};
% \legend{Top 1, Top 2, Top 3}
% \end{axis}
% \end{tikzpicture}
% \caption{reuslts!}
% \label{fig:results}
% \end{figure}

\makeatletter
\newcommand\resetstackedplots{
\makeatletter
\pgfplots@stacked@isfirstplottrue
\makeatother
\addplot [forget plot,draw=none] coordinates{
  (baseline,0) (spacer1,0)
  (ocaml,0) (mycroft,0) (sherrloc,0) (spacer2,0)
  (op+type+size/linear,0)
  (op+type+size/decision-tree,0)
  (op+type+size/hidden-10,0)
  (op+type+size/hidden-250,0)
};
}
\makeatother

\begin{figure}[ht]
\centering
\begin{tikzpicture}
\begin{axis}[
  ybar stacked,
  width=14cm,
  height=8cm,
  title={Comparison of Type-Error Localization Techniques},
  ylabel={Accuracy},
  bar width=0.5cm,
  ymin=0.2,
  ymax=1,
  yticklabel={\pgfmathparse{\tick*100}\pgfmathprintnumber{\pgfmathresult}\,\%},
  ytick style={draw=none},
  ymajorgrids = true,
  symbolic x coords={baseline, spacer1, ocaml, mycroft, sherrloc, spacer2,
                     op+type+size/linear,
                     op+type+size/decision-tree,
                     op+type+size/hidden-10,
                     op+type+size/hidden-250},
  % enlarge x limits=0.5,
  xtick=data,
  xtick style={draw=none},
  xticklabels={\baseline, \ocaml, \mycroft, \sherrloc,
               \linear, \dectree, \hiddenT, \hiddenTHF},
  x tick label style={rotate=45},
  reverse legend,
  transpose legend,
  legend style={legend pos = north west, legend columns=4},
]
\addplot[draw=black, fill=blue1, bar shift=-.25cm] table[x=tool, y=top-1] {\SpringBench};
\addplot[draw=black, fill=blue2, bar shift=-.25cm] table[x=tool, y expr=\thisrow{top-2} - \thisrow{top-1}] {\SpringBench};
\addplot[draw=black, fill=blue3, bar shift=-.25cm] table[x=tool, y expr=\thisrow{top-3} - \thisrow{top-2}] {\SpringBench};
\addlegendentry{Top 1}
\addlegendentry{Top 2}
\addlegendentry{Top 3}
\addlegendimage{empty legend}
\addlegendentry{FA15}

\resetstackedplots

\addplot[draw=black, fill=blue1, bar shift=.25cm] table[x=tool, y=top-1] {\FallBench};
\addplot[draw=black, fill=blue2, bar shift=.25cm] table[x=tool, y expr=\thisrow{top-2} - \thisrow{top-1}] {\FallBench};
\addplot[draw=black, fill=blue3, bar shift=.25cm] table[x=tool, y expr=\thisrow{top-3} - \thisrow{top-2}] {\FallBench};
\addlegendentry{Top 1}
\addlegendentry{Top 2}
\addlegendentry{Top 3}
\addlegendentry{SP14}
\addlegendimage{empty legend}
%\legend{Top 1, Top 2, Top 3}
\end{axis}
\end{tikzpicture}
\caption{reuslts!}
\label{fig:results}
\end{figure}






\subsection{Qualitative Evaluation}
\label{sec:qualitative}

Next, we present a \emph{qualitative} evaluation that compares the
predictions made by our classifiers with those of \sherrloc.
%
In particular, we demonstrate, with a series of example programs from
our student dataset, how our classifiers are able to use past student
mistakes to make more accurate predictions of future fixes.
%
For each example, we provide
%
(1) the code,
%
(2) \sherrloc's prediction (in \textbf{bold}), and
%
(3) the prediction of our Decision Tree (\underline{underlined}).
%
We choose the Decision Tree classifier for this section as its model
is more easily interpreted than the MLP.
%
We will also attempt, for each program, to explain \emph{why} the
Decision Tree made its prediction, by analyzing the paths induced
by the programs.

\paragraph{Extracting the Digits of an Integer}
% data/sp14/1655.ml
Our first program is a simple recursive function |digitsOfInt| that
extracts the digits of an |int|.
%
\begin{ecode}
  let rec digitsOfInt n =
    if n <= 0 then
      []
    else
      [n mod 10] @ __[ ___=digitsOfInt (n / 10)=___ ]__
\end{ecode}
%
Unfortunately, the student has decided to wrap the recursive call to
|digitsOfInt| with a list literal, even though |digitsOfInt| already
returns an |int list|.
%
Thus, the list literal is inferred to have type |int list list|, which
is incompatible with the |int list| on the left of the |@| (list append)
operator.
%
Both \sherrloc and the \ocaml compiler blame the recursive call for
returning a |int list| rather than |int|, but the recursive call is
actually correct!
%
As our Decision Tree correctly points out, the fault lies with the list
literal \emph{surrounding} the recursive call, remove it and the type
error disappears.
%
\ES{TODO: WHY?}

\paragraph{Padding a list}
% data/sp14/0306.ml
Our next program, |padZero|, is given two |int list|s as input, and must
left-pad the shorter one with enough zeros that the two output lists
have equal length.

The student first defines a helper function |clone|, which takes an
input element |x| and an |int| |n|, and returns a list containing |n|
|x|s.
%
\begin{ecode}
  let rec clone x n =
    if n <= 0 then
      []
    else
      x :: clone x (n - 1)
\end{ecode}
%
Then she defines |padZero| with a simple branch that determines which
list is shorter, followed by a |clone| with the appropriate number of
zeros.
%
\lstset{firstnumber=last}
\begin{ecode}
  let padZero l1 l2 =
    let n = List.length l1 - List.length l2 in
    if n < 0 then
      (clone 0 ((-1) * n) @ l2, l2)
    else
      (l1, _=clone 0 n=___ :: l2__)
\end{ecode}
\lstset{firstnumber=1}
%
Alas, our student has accidentally used the |::| operator rather than
the |@| operator in the |else| branch.
%
\sherrloc and \ocaml correctly determine that she cannot cons the
|int list| returned by |clone| onto |l2|, which is another |int list|,
but they decide to \emph{blame} the call to |clone|, while our Decision
Tree correctly blames the |::| operator.
%
\ES{TODO: WHY? (might be too similar to previous?)}


% ES: decision tree gets this one wrong, may want to find something else
% \begin{ecode}
%   let rec sepConcat sep sl =
%     match sl with
%     | [] -> ""
%     | h::t ->
%         let f a x = a ^ (sep ^ x) in
%         List.fold_left f h t

%   let stringOfList f l = sepConcat "; " __[ "["; ___=List.map f l=___; "]" ]__
% \end{ecode}

\paragraph{Computing the Fixed Point of a Function}
% data/sp14/0941.ml
Finally, our students must write a |fixpoint| function that computes the
fixed point of a given function |f|, starting from an initial value |b|.
%
As a hint we first have them write a |wwhile| function that performs the
functional equivalent of a \texttt{while}-loop, repeatedly passing a function's
output back in until it receives a (boolean) signal to stop.
%
\begin{ecode}
  let rec wwhile (f, b) =
    match f b with
    | (x, false) -> x
    | (x, true)  -> wwhile (f, x)
\end{ecode}
\lstset{firstnumber=last}
%
The |fixpoint| function can then be written as a clever instantiation of
the arguments to |wwhile|.
%
\begin{ecode}
  let fixpoint (f, b) =
    let g = __let bb = f b in ___=(bb, (bb = b))=_ in
    wwhile (g, b)
\end{ecode}
\lstset{firstnumber=1}
%
Sadly, our student has forgotten that |g| should itself be a function.
%
As a result, she passes a pair of a \emph{pair} and a starting value to
|wwhile|, rather than a pair of a \emph{function} and a starting value.
%
\sherrloc deduces that the call to |wwhile| is likely correct (\ocaml
actually blames the use of |g|), but identifies the construction of the
pair inside |g| as the most likely culprit, while our Decision Tree
correctly identifies the \emph{definition} of |g| as the source of the
error.
\ES{TODO: WHY?}


\subsubsection{Failed Predictions}
\label{sec:failed-predictions}
Of course, our classifiers are sometimes wrong, we focus next on programs
where our classifier makes an incorrect prediction.

\paragraph{Constructing a List of Duplicates}
% data/sp14/0148.ml
Consider the following implementation of the |clone| function from
before.
%
\begin{ecode}
  let rec clone x n =
    let loop acc n =
      if n <= 0 then
        acc
      else
        clone ==([==_=x=_==] @ acc)== (n - 1) in
    loop [] n
\end{ecode}
% \ES{TODO: our 2nd prediction matches \sherrloc and \ocaml (occurs check), correct fix is to replace recursive call to clone with loop}
%
The student has defined a helper function |loop| with an accumulator
|acc|, likely meant to call itself tail-recursively.
%
Unfortunately, she has called the top-level function |clone| rather than
|loop| in the |else| branch, this induces a cyclic constraint |'a = 'a list|
for the |x| argument to |clone|.

Our classifier incorrectly predicts that the use of |x| in the recursive
call is the most likely source of the error.
\ES{TOOD: WHY??}
%
Our second prediction coincides with \sherrloc (and \ocaml), blaming the
the first argument to |clone|.
%
This is also incorrect, but may be more helpful than our first
prediction --- if our student decides that she has certainly provided
the correct \emph{argument}, an alternative explanation is that
perhaps she has called the wrong \emph{function}.
%
Our final prediction is the entire body of |clone|, which is hardly
helpful.

\paragraph{Currying Considered Harmful?}
% data/sp14/0887.ml
Our next example is another ill-fated attempt at |clone|.
%
\begin{ecode}
  let rec clone x n =
    let rec loop x n acc =
      if n < 0 then
        acc
      else
        ==loop== __(x, (n - 1), (x :: acc))__ in
    loop (x, n, [])
\end{ecode}
The issue here is that \ocaml functions are \emph{curried} by default
--- \ie they take their arguments one at a time --- but our student has
called the inner |loop| with all three arguments in a tuple.
%
Many experienced functional programmers would choose to keep |loop|
curried and rewrite the calls, however our student decides instead to
\emph{uncurry} |loop|, making it take a tuple of arguments.
%
\sherrloc blames the recursive call to |loop| while our classifier
blames the tuple of arguments --- a reasonable suggestion, but alas not
the answer the student was looking for.
\ES{WHY??}
%

\paragraph{Composing Functions}
% data/sp14/2269.ml
Next, let us consider the |pipe| function that composes a list of
functions, \ie given a list of functions |[f;g;h]|, |pipe| produces the
function |fun x -> h (g (f x))|.
%
\begin{ecode}
  let pipe fs =
    let f a x y = ==y== __(a y)__ in
    let base x = x in
    List.fold_left f base fs
\end{ecode}
%
The error in our student's code is that she has applied |y| rather than
|x| to the result of |a y|.
%
\sherrloc correctly blames the first occurrence of |y|, while our
classifier (incorrectly) blames the application |a y| (\ocaml blames
the occurrence of |base| on line 4).
%
\ES{TODO: WHY?}


%%% Local Variables:
%%% mode: latex
%%% TeX-master: "main"
%%% End:


%%% Local Variables:
%%% mode: latex
%%% TeX-master: "main"
%%% End:
