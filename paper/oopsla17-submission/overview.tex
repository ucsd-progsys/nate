\section{Overview}
\label{sec:overview}

We start with an overview of our approach to localizing type errors by
learning a model of the mistakes programmers actually make.
%
We formulate the problem of type error localization as a
\emph{supervised classification} problem.
%
A \emph{classification} problem entails learning a function that maps
inputs to a discrete set of output \emph{labels}, in contrast to
\emph{regression} where the output is typically a real number.
%
In a \emph{supervised} learning problem one is given a \emph{training}
set where the inputs and labels are known, and the task is to learn a
function that accurately maps the inputs to outputs and
\emph{generalizes} to new, yet-unseen inputs.

The key technical challenges we address are:
%
(1) formulating type error localization as a classification problem,
%
(2) generating correct labels for supervised learning, and
%
(3) presenting the predictions of the classifier in an intuitive manner.

\paragraph{\textbf{Challenge (1):} How to represent programs?}


\paragraph{\textbf{Solution:} Use a \textbf{set} of feature vectors}

\paragraph{\textbf{Challenge (2):} What is the ``correct'' fix?}

\paragraph{\textbf{Solution:} Use the students fixes}

\paragraph{\textbf{Challenge (3):} What should be presented to the user?}

\paragraph{\textbf{Solution:} Rank predictions and present most likely}




%%% Local Variables:
%%% mode: latex
%%% TeX-master: "main"
%%% End:
