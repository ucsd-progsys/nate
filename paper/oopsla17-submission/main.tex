%% For double-blind review submission
\documentclass[acmlarge,review,anonymous]{acmart}\settopmatter{printfolios=true}
%% For single-blind review submission
%\documentclass[acmlarge,review]{acmart}\settopmatter{printfolios=true}
%% For final camera-ready submission
%\documentclass[acmlarge]{acmart}\settopmatter{}

%% Note: Authors migrating a paper from PACMPL format to traditional
%% SIGPLAN proceedings format should change 'acmlarge' to
%% 'sigplan,10pt'.


%% Some recommended packages.
\usepackage{booktabs}   %% For formal tables:
                        %% http://ctan.org/pkg/booktabs
\usepackage{subcaption} %% For complex figures with subfigures/subcaptions
                        %% http://ctan.org/pkg/subcaption

\usepackage{amsmath}
\usepackage{amssymb}
\usepackage{xspace}
\usepackage{booktabs}
\usepackage{listings}
\lstset{
  language=Caml,
  basicstyle=\ttfamily
}
\usepackage{commands}
\def\sectionautorefname{\S}
\def\subsectionautorefname{\S}

\synctex=1

\makeatletter\if@ACM@journal\makeatother
%% Journal information (used by PACMPL format)
%% Supplied to authors by publisher for camera-ready submission
\acmJournal{PACMPL}
\acmVolume{1}
\acmNumber{1}
\acmArticle{1}
\acmYear{2017}
\acmMonth{1}
\acmDOI{10.1145/nnnnnnn.nnnnnnn}
\startPage{1}
\else\makeatother
%% Conference information (used by SIGPLAN proceedings format)
%% Supplied to authors by publisher for camera-ready submission
\acmConference[PL'17]{ACM SIGPLAN Conference on Programming Languages}{January 01--03, 2017}{New York, NY, USA}
\acmYear{2017}
\acmISBN{978-x-xxxx-xxxx-x/YY/MM}
\acmDOI{10.1145/nnnnnnn.nnnnnnn}
\startPage{1}
\fi


%% Copyright information
%% Supplied to authors (based on authors' rights management selection;
%% see authors.acm.org) by publisher for camera-ready submission
\setcopyright{none}             %% For review submission
%\setcopyright{acmcopyright}
%\setcopyright{acmlicensed}
%\setcopyright{rightsretained}
%\copyrightyear{2017}           %% If different from \acmYear


%% Bibliography style
\bibliographystyle{ACM-Reference-Format}
%% Citation style
%% Note: author/year citations are required for papers published as an
%% issue of PACMPL.
\citestyle{acmauthoryear}   %% For author/year citations



\begin{document}

%% Title information
%% [Short Title] is optional; when present, will be used in header
%% instead of Full Title.
\title[Short Title]{Learning to Blame}

% \titlenote{with title note}             %% \titlenote is optional;
%                                         %% can be repeated if necessary;
%                                         %% contents suppressed with 'anonymous'
%% \subtitle is optional
\subtitle{Improving Type Errors with Small Data}
% \subtitlenote{with subtitle note}       %% \subtitlenote is optional;
%                                         %% can be repeated if necessary;
%                                         %% contents suppressed with 'anonymous'


%% Author information
%% Contents and number of authors suppressed with 'anonymous'.
%% Each author should be introduced by \author, followed by
%% \authornote (optional), \orcid (optional), \affiliation, and
%% \email.
%% An author may have multiple affiliations and/or emails; repeat the
%% appropriate command.
%% Many elements are not rendered, but should be provided for metadata
%% extraction tools.

%% Author with single affiliation.
\author{Eric L. Seidel}
% \authornote{with author1 note}          %% \authornote is optional;
%                                         %% can be repeated if necessary
\orcid{0000-0002-2529-7790}             %% \orcid is optional
\affiliation{
  \department{Department of Computer Science}              %% \department is recommended
  \institution{UC San Diego}            %% \institution is required
  \city{La Jolla}
  \state{CA}
  \country{USA}
}
\email{eseidel@cs.ucsd.edu}          %% \email is recommended

\author{Huma Sibghat}
% \authornote{with author1 note}          %% \authornote is optional;
%                                         %% can be repeated if necessary
% \orcid{nnnn-nnnn-nnnn-nnnn}             %% \orcid is optional
\affiliation{
  \department{Department of Computer Science}              %% \department is recommended
  \institution{UC San Diego}            %% \institution is required
  \city{La Jolla}
  \state{CA}
  \country{USA}
}
\email{hsibghat@cs.ucsd.edu}          %% \email is recommended

\author{Kamalika Chaudhuri}
% \authornote{with author1 note}          %% \authornote is optional;
%                                         %% can be repeated if necessary
% \orcid{nnnn-nnnn-nnnn-nnnn}             %% \orcid is optional
\affiliation{
  \department{Department of Computer Science}              %% \department is recommended
  \institution{UC San Diego}            %% \institution is required
  \city{La Jolla}
  \state{CA}
  \country{USA}
}
\email{kamalika@cs.ucsd.edu}          %% \email is recommended

\author{Ranjit Jhala}
% \authornote{with author1 note}          %% \authornote is optional;
%                                         %% can be repeated if necessary
% \orcid{nnnn-nnnn-nnnn-nnnn}             %% \orcid is optional
\affiliation{
  \department{Department of Computer Science}              %% \department is recommended
  \institution{UC San Diego}            %% \institution is required
  \city{La Jolla}
  \state{CA}
  \country{USA}
}
\email{jhala@cs.ucsd.edu}          %% \email is recommended

\author{Westley Weimer}
% \authornote{with author1 note}          %% \authornote is optional;
%                                         %% can be repeated if necessary
% \orcid{nnnn-nnnn-nnnn-nnnn}             %% \orcid is optional
\affiliation{
  \department{Department of Computer Science}              %% \department is recommended
  \institution{University of Virginia}            %% \institution is required
  % \city{La Jolla}
  % \state{CA}
  \country{USA}
}
\email{weimer@cs.virginia.edu}          %% \email is recommended


%% Paper note
%% The \thanks command may be used to create a "paper note" ---
%% similar to a title note or an author note, but not explicitly
%% associated with a particular element.  It will appear immediately
%% above the permission/copyright statement.
\thanks{with paper note}                %% \thanks is optional
                                        %% can be repeated if necesary
                                        %% contents suppressed with 'anonymous'


%% Abstract
%% Note: \begin{abstract}...\end{abstract} environment must come
%% before \maketitle command
\begin{abstract}
Text of abstract \ldots.
\end{abstract}


%% 2012 ACM Computing Classification System (CSS) concepts
%% Generate at 'http://dl.acm.org/ccs/ccs.cfm'.
\begin{CCSXML}
<ccs2012>
<concept>
<concept_id>10011007.10011006.10011008</concept_id>
<concept_desc>Software and its engineering~General programming languages</concept_desc>
<concept_significance>500</concept_significance>
</concept>
<concept>
<concept_id>10003752.10003790.10011740</concept_id>
<concept_desc>Theory of computation~Type theory</concept_desc>
<concept_significance>500</concept_significance>
</concept>
<concept>
<concept_id>10010147.10010257</concept_id>
<concept_desc>Computing methodologies~Machine learning</concept_desc>
<concept_significance>300</concept_significance>
</concept>
</ccs2012>
\end{CCSXML}

\ccsdesc[500]{Software and its engineering~General programming languages}
\ccsdesc[500]{Theory of computation~Type theory}
\ccsdesc[300]{Computing methodologies~Machine learning}
%% End of generated code


%% Keywords
%% comma separated list
\keywords{keyword1, keyword2, keyword3}  %% \keywords is optional


%% \maketitle
%% Note: \maketitle command must come after title commands, author
%% commands, abstract environment, Computing Classification System
%% environment and commands, and keywords command.
\maketitle


\section{Introduction}
\label{sec:introduction}

\subsection{Contributions}
\label{sec:contributions}
Thus, we identify the following novel contributions of our work.
%
\begin{enumerate}
\item An large-scale evaluation of state-of-the-art type error
  localization techniques on over 4,500 student programs.
  %
  We collected a time-series of \ocaml programs from our students, and
  use their subsequent \emph{fixes} to ill-typed programs as
  \emph{oracles} for the location of the error.
  %
  This is the most extensive evaluation of the accuracy of type error
  localization techniques that we are aware of.
  % based not on expert \emph{opinion}, but
  % on the eventual \emph{fix} implemented by a novice user.
  %
  % \ES{need to explain \emph{why} this is superior..}
\item A machine learning approach to \emph{modeling} (novice) type
  errors that can outperform the state-of-the-art by 15--30\% in
  localizing errors.
  %
  Our classifiers can be trained on a relatively small amount of data --- we
  used a single class of around 50 students --- and appear to
  generalize to other instances of the same class.
  %
  \ES{and hopefully completely separate classes!}
  %
  In contrast to many approaches to \emph{fault localization} from the
  software engineering community, our model does not use any linguistic
  modeling techniques, \eg \emph{n-grams} over the token stream; rather,
  it relies entirely on features of the abstract syntax tree and a
  partial typing derivation.
  %
  \ES{if there's time, investigate adding Wes' n-gram model}
\end{enumerate}

%%% Local Variables:
%%% mode: latex
%%% TeX-master: "main"
%%% End:

\section{Overview}
\label{sec:overview}

We start with an overview of our approach to localizing type errors by
learning a model of the mistakes programmers actually make.
%
We formulate the problem of type error localization as a
\emph{supervised classification} problem.
%
A \emph{classification} problem entails learning a function that maps
inputs to a discrete set of output \emph{labels}, in contrast to
\emph{regression} where the output is typically a real number.
%
In a \emph{supervised} learning problem one is given a \emph{training}
set where the inputs and labels are known, and the task is to learn a
function that accurately maps the inputs to outputs and
\emph{generalizes} to new, yet-unseen inputs.

The key technical challenges we address are:
%
(1) formulating type error localization as a classification problem,
%
(2) generating correct labels for supervised learning, and
%
(3) presenting the predictions of the classifier in an intuitive manner.

\paragraph{\textbf{Challenge (1):} How to represent programs?}


\paragraph{\textbf{Solution:} Use a \textbf{set} of feature vectors}

\paragraph{\textbf{Challenge (2):} What is the ``correct'' fix?}

\paragraph{\textbf{Solution:} Use the students fixes}

\paragraph{\textbf{Challenge (3):} What should be presented to the user?}

\paragraph{\textbf{Solution:} Rank predictions and present most likely}




%%% Local Variables:
%%% mode: latex
%%% TeX-master: "main"
%%% End:

\section{Learning to Blame}
\label{sec:learning}

\subsection{Syntax}
\label{sec:syntax}
\begin{figure}
$$
\begin{array}{rrcl}
\emphbf{Expressions}
  & e & ::=    & x \spmid \efun{x}{e} \spmid \eapp{e}{e} \spmid \elet{x}{e}{e} \\
  &   & \spmid & n \spmid \eplus{e}{e}\\
  &   & \spmid & b \spmid \eif{e}{e}{e} \\
  &   & \spmid & \epair{e}{e} \spmid \epcase{e}{x}{x}{e} \\
  &   & \spmid & \enil \spmid \econs{e}{e} \spmid \ecase{e}{e}{x}{x}{e} \\[0.05in]

\emphbf{Integers}
  & n & ::= &  0, 1, -1, \ldots \\[0.05in]

\emphbf{Booleans}
  & b & ::= &  \etrue \spmid \efalse \\[0.05in]

\emphbf{Types}
  & t & ::= & \alpha \spmid \tbool \spmid \tint \spmid \tfun{t}{t} \spmid \tprod{t}{t} \spmid \tlist{t} \\[0.05in]
\end{array}
$$
\caption{Syntax of \lang}
\label{fig:syntax}
\end{figure}

%
\autoref{fig:syntax} describes the syntax of \lang, a simple lambda
calculus with integers, booleans, pairs, and lists.

\subsection{Features}
\label{sec:features}
The first issue we must tackle is formulating our learning task in
machine learning terms.
%
We are given expressions $e$, but the learning algorithms expect to work
with \emph{feature vectors} --- vectors of real numbers, where each
column describes a particular feature of the input.
%
Thus, our first task is to convert expressions to feature vectors.

We choose to model a program as a \emph{set} of feature vectors, where
each element corresponds a sub-expression in the program. We group the
features into the following categories.

\paragraph{Local syntactic features}
These features describe the syntactic category of each sub-expression
$e$.
%
In other words, for each production of $e$ in \autoref{fig:syntax} we
introduce a feature that is enabled (set to $1$) \emph{iff} the
sub-expression was built with that production.

\paragraph{Contextual syntactic features}
These are like local syntactic features, but lifted to describe the
parent and children of the current sub-expression.
%
If a particular $e$ does not have children (\eg a variable $x$) or a
parent (\ie the root expression), we leave the corresponding features
disabled (set to $0$).
%
This gives us a notion of the \emph{context} in which an expression
occurs, similar to the \emph{n-grams} commonly found in linguistic
models.

Instead of just describing the immediate context, we could describe
whether a particular syntax element occurs in the neighboring
sub-expressions (or even a count of how many times it occurs).
%
Such fuzzier notions of context may enable increased precision in the
model, but they also introduce opportunities for \emph{overfitting} ---
where the model memorizes particular inputs rather than learning general
patterns.
%
We will investigate (\ES{maybe..}) the impact of these alternatives
in \autoref{sec:evaluation}.

\paragraph{Expression size}
We also add a feature representing the \emph{size} of each expression,
\ie how many sub-expressions does it contain?
%
This allows the model to learn that, \eg, expressions closer to the
leaves are more likely to be blamed than expressions closer to the root.

\paragraph{Typing features}
A natural way of summarizing the context in which an expression occurs
is with \emph{types}.
%
A difficulty that arises here is that, due to the parametric type
constructors $\tfun{\cdot}{\cdot}$, $\tprod{\cdot}{\cdot}$, and
$\tlist{\cdot}$, there is an \emph{infinite} set of possible types ---
but we must have a \emph{finite} set of features.
%
Thus, we add features for each type \emph{constructor} that describe
whether a given type \emph{mentions} the type constructor.
%
For example, the type $\tint$ would only enable the $\tint$ feature,
while the type $\tfun{\tint}{\tbool}$ would enable the
$\tfun{\cdot}{\cdot}$, $\tint$, and $\tbool$ features.
%
We add these features for parent and child expressions to summarize the
context, but also for the current expression, as the type of an
expression is not always clear \emph{syntactically} (\eg for variables
and applications).
\ES{note the possibility of traversal bias from typing features}

\paragraph{Type error slice}
Finally, we would like the model to be able to distinguish between
changes that could fix the error, and changes that
\emph{cannot possibly} fix the error.
%
Thus, we compute a minimal type error \emph{slice} for the program
(\ie the set of expressions that contribute to the error), and add a
feature that is enabled \emph{iff} an expression is part of the slice.
%
If the program contains multiple type errors, we will compute a minimal
slice for each error.

\subsection{Labels}
\label{sec:labels}

\subsection{Models}
\label{sec:models}





%%% Local Variables:
%%% mode: latex
%%% TeX-master: "main"
%%% End:

\section{Evaluation}
\label{sec:evaluation}

\pgfplotstableset{col sep=comma}

\pgfplotstableread{../../data/sp14/op+type+size/linear/results.csv}{\FeatureLinearBench}
\pgfplotstablevertcat{\FeatureLinearBench}{../../data/sp14/op+context+type+size/linear/results.csv}
\pgfplotstablevertcat{\FeatureLinearBench}{../../data/sp14/op+context-has+type+size/linear/results.csv}
\pgfplotstablevertcat{\FeatureLinearBench}{../../data/sp14/op+context-count+type+size/linear/results.csv}
\pgfplotstableread{../../data/sp14/op+type+size/hidden-10/results.csv}{\FeatureHiddenTBench}
\pgfplotstablevertcat{\FeatureHiddenTBench}{../../data/sp14/op+context+type+size/hidden-10/results.csv}
\pgfplotstablevertcat{\FeatureHiddenTBench}{../../data/sp14/op+context-has+type+size/hidden-10/results.csv}
\pgfplotstablevertcat{\FeatureHiddenTBench}{../../data/sp14/op+context-count+type+size/hidden-10/results.csv}
\pgfplotstableread{../../data/sp14/op+type+size/hidden-500/results.csv}{\FeatureHiddenFHBench}
\pgfplotstablevertcat{\FeatureHiddenFHBench}{../../data/sp14/op+context+type+size/hidden-500/results.csv}
\pgfplotstablevertcat{\FeatureHiddenFHBench}{../../data/sp14/op+context-has+type+size/hidden-500/results.csv}
\pgfplotstablevertcat{\FeatureHiddenFHBench}{../../data/sp14/op+context-count+type+size/hidden-500/results.csv}

\pgfplotstableread{../../data/sp14/op+type+size/hidden-10/results.csv}{\HiddenBench}
\pgfplotstablevertcat{\HiddenBench}{../../data/sp14/op+type+size/hidden-25/results.csv}
\pgfplotstablevertcat{\HiddenBench}{../../data/sp14/op+type+size/hidden-50/results.csv}
\pgfplotstablevertcat{\HiddenBench}{../../data/sp14/op+type+size/hidden-100/results.csv}
\pgfplotstablevertcat{\HiddenBench}{../../data/sp14/op+type+size/hidden-250/results.csv}
\pgfplotstablevertcat{\HiddenBench}{../../data/sp14/op+type+size/hidden-500/results.csv}
% \pgfplotstablevertcat{\HiddenBench}{../../data/sp14/op+context-count+type+size/hidden-10/results.csv}
% \pgfplotstablevertcat{\HiddenBench}{../../data/sp14/op+context-count+type+size/hidden-25/results.csv}
% \pgfplotstablevertcat{\HiddenBench}{../../data/sp14/op+context-count+type+size/hidden-50/results.csv}
% \pgfplotstablevertcat{\HiddenBench}{../../data/sp14/op+context-count+type+size/hidden-100/results.csv}
% \pgfplotstablevertcat{\HiddenBench}{../../data/sp14/op+context-count+type+size/hidden-250/results.csv}
% \pgfplotstablevertcat{\HiddenBench}{../../data/sp14/op+context-count+type+size/hidden-500/results.csv}

\pgfplotstableread{../../data/sp14/baseline.csv}{\SpringBench}
\pgfplotstablevertcat{\SpringBench}{../../data/sp14/ocaml/results.csv}
\pgfplotstablevertcat{\SpringBench}{../../data/sp14/mycroft/results.csv}
\pgfplotstablevertcat{\SpringBench}{../../data/sp14/sherrloc/results.csv}
\pgfplotstablevertcat{\SpringBench}{../../data/sp14/op+type+size/linear/results.csv}
\pgfplotstablevertcat{\SpringBench}{../../data/sp14/op+type+size/decision-tree/results.csv}
\pgfplotstablevertcat{\SpringBench}{../../data/sp14/op+type+size/hidden-10/results.csv}
\pgfplotstablevertcat{\SpringBench}{../../data/sp14/op+type+size/hidden-250/results.csv}

\pgfplotstableread{../../data/fa15/baseline.csv}{\FallBench}
\pgfplotstablevertcat{\FallBench}{../../data/fa15/ocaml/results.csv}
\pgfplotstablevertcat{\FallBench}{../../data/fa15/mycroft/results.csv}
\pgfplotstablevertcat{\FallBench}{../../data/fa15/sherrloc/results.csv}
\pgfplotstablevertcat{\FallBench}{../../data/fa15/op+type+size/linear/results.csv}
\pgfplotstablevertcat{\FallBench}{../../data/fa15/op+type+size/decision-tree/results.csv}
\pgfplotstablevertcat{\FallBench}{../../data/fa15/op+type+size/hidden-10/results.csv}
\pgfplotstablevertcat{\FallBench}{../../data/fa15/op+type+size/hidden-250/results.csv}


\subsection{Experimental Setup}
\label{sec:experimental-setup}

\subsection{Type Error Slice}
\label{sec:type-error-slice}
The \InSlice feature should be highly predictive --- a fix must change
at least one expression in the type-error slice.
%
Thus, our first experiment seeks to quantify the impact of \InSlice by
comparing the accuracy of a linear model on three sets of features.
%
\begin{enumerate}
\item A baseline with only local syntactic features.
\item extends (1) with \InSlice.
\item discards samples from (1) where \InSlice is \emph{disabled}.
\end{enumerate}
%
The key difference between (2) and (3) is that a model for (2) must
\emph{learn} that \InSlice is a strong predictor.
%
In contrast, a model for (3) must only learn about the syntactic
features, the decision to discard samples where \InSlice is disabled has
already been made by a human.
%
This has a few additional advantages: it reduces the search space by a
factor of 7 on average, and it guarantees that any prediction made by
the classifier can fix the type error.
%
We expect that (2) will perform better than (1) as it has more
information, and that (3) will perform better than (2) as it does not
have to learn the importance of \InSlice.

We tested our hypothesis with a linear model cross-validated ($k=10$)
over the combined SP14/FA15 dataset. We used a learning rate
$\alpha=0.001$, L2 regularization rate $\lambda=0.001$, and mini-batch
size of 200. We trained for a single epoch on feature sets (1) and
(2), and for 8 epochs on (3), so that the total number of training samples
would be roughly equal for each feature set.

\begin{table}[ht]
  \centering
  \begin{tabular}{lrrrr}
    \toprule
    Feature Set  & Top-1  & Top-2  & Top-3  & Recall \\
    \midrule
    Local Syntax & 23.6\% & 42.6\% & 56.3\% & 19.6\% \\
    With Slice   & 46.4\% & 65.0\% & 75.4\% & 30.4\% \\
    Only Slice   & 54.9\% & 71.4\% & 82.5\% & 57.6\% \\
    \bottomrule
  \end{tabular}
  \caption{
    Impact of type-error slice on accuracy.
    \ES{TODO: expand caption}
    \ES{TODO: load these numbers from CSV}
  }\label{tab:type-error-slice}
\end{table}

\autoref{tab:type-error-slice} shows the results of our experiment.
%
As expected the baseline performs the worst, with a mere 23.6\% Top-1
accuracy.
%
Adding \InSlice improves the results substantially with a 46.4\% Top-1
accuracy, demonstrating the importance of a minimal error slice.
%
However, filtering out expressions that are not part of the slice
\emph{further} improves the results to 54.9\% Top-1 accuracy.
%
Clearly, some decisions are too important to be left to a machine.

Note also the jump in Recall when we filter out expressions that are not
part of the error slice.
%
Reducing the search space not only improves our chances of making a
single correct prediction, it also allows us to make \emph{multiple}
correct predictions per program.

Given the decisive benefits of filtering out expressions that do not
belong to the type-error slice, we will assume going forward that
\emph{all} data sets have been filtered.

\subsection{Contextual Features}
\label{sec:contextual-features}

Next, we will investigate the relative impact of the other three classes
of features discussed in \autoref{sec:features}.
%
For this we consider again a baseline of only local syntactic features,
extended by each combination of
%
(1) expression size,
(2) contextual syntactic features, and
(3) typing features.
%
As before we perform a 10-fold cross-validation with a learning rate and
L2 regularization rate of 0.001 and a mini-batch size of 200, but we
train for a full 10 epochs.

\begin{table}[ht]
  \centering
  \begin{tabular}{lrrrr}
    \toprule
    Feature Set                 & Top-1  & Top-2  & Top-3  & Recall \\
    \midrule
    Local Syntax                & 55.0\% & 71.6\% & 82.6\% & 57.7\% \\
    \midrule
    + Size                      & 55.8\% & 72.8\% & 82.5\% & 57.3\% \\
    + Contextual Syntax         & 59.9\% & 77.7\% & 86.4\% & 63.0\% \\
    + Types                     & 62.2\% & 77.7\% & 85.7\% & 62.2\% \\
    \midrule
    + Size + Contextual Syntax  & 60.2\% & 77.9\% & 86.0\% & 62.5\% \\
    + Size + Types              & 62.0\% & 78.2\% & 85.6\% & 62.3\% \\
    + Contextual Syntax + Types & 63.2\% & 80.3\% & 87.9\% & 65.4\% \\
    \midrule
    + All                       & 62.3\% & 80.0\% & 88.0\% & 65.4\% \\
    \bottomrule
  \end{tabular}
  \caption{
    Impact of contextual features on accuracy.
    \ES{TODO: expand caption}
    \ES{TODO: load these numbers from CSV}
  }\label{tab:type-error-slice}
\end{table}


\definecolor{blue1}{HTML}{DEEBF7}
\definecolor{blue2}{HTML}{9ECAE1}
\definecolor{blue3}{HTML}{3182BD}

% \begin{figure}[ht]
% \centering
% \begin{tikzpicture}
% \begin{axis}[
%   % ybar stacked,
%   width=12cm,
%   height=8cm,
%   title={Impact of Feature Set on Accuracy},
%   ylabel={Accuracy},
%   %ymin=0.2,
%   ymax=1,
%   yticklabel={\pgfmathparse{\tick*100}\pgfmathprintnumber{\pgfmathresult}\,\%},
%   ytick style={draw=none},
%   ymajorgrids = true,
%   symbolic x coords={op+type+size, op+context+type+size, op+context-has+type+size, op+context-count+type+size},
%   % enlarge x limits=0.25,
%   xtick=data,
%   xtick style={draw=none},
%   xticklabels={Type, Context-Is, Context-Has, Context-Count},
%   x tick label style={rotate=45},
%   reverse legend,
%   transpose legend,
%   legend style={legend pos = outer north east, legend columns=4},
% ]
% % \addplot[draw=black, fill=blue1] table[x=tool, y=top-1] {\HiddenBench};
% % \addplot[draw=black, fill=blue2] table[x=tool, y expr=\thisrow{top-2} - \thisrow{top-1}] {\HiddenBench};
% % \addplot[draw=black, fill=blue3] table[x=tool, y expr=\thisrow{top-3} - \thisrow{top-2}] {\HiddenBench};

% \addplot[mark options={fill=blue1, scale=1.5}, mark=square*]
%   table[x=features, y=top-1] {\FeatureHiddenFHBench};
% \addplot[mark options={fill=blue2, scale=1.5}, mark=square*]
%   table[x=features, y=top-2] {\FeatureHiddenFHBench};
% \addplot[mark options={fill=blue3, scale=1.5}, mark=square*]
%   table[x=features, y=top-3] {\FeatureHiddenFHBench};
% \addlegendentry{Top 1}
% \addlegendentry{Top 2}
% \addlegendentry{Top 3}
% \addlegendimage{empty legend}
% \addlegendentry{\hiddenFH}

% \addplot[mark options={fill=blue1, scale=1.5}, mark=*]
%   table[x=features, y=top-1] {\FeatureLinearBench};
% \addplot[mark options={fill=blue2, scale=1.5}, mark=*]
%   table[x=features, y=top-2] {\FeatureLinearBench};
% \addplot[mark options={fill=blue3, scale=1.5}, mark=*]
%   table[x=features, y=top-3] {\FeatureLinearBench};
% \addlegendentry{Top 1}
% \addlegendentry{Top 2}
% \addlegendentry{Top 3}
% \addlegendimage{empty legend}
% \addlegendentry{\linear}

% \end{axis}
% \end{tikzpicture}
% \caption{reuslts!}
% \label{fig:results}
% \end{figure}

% \begin{figure}[ht]
% \centering
% \begin{tikzpicture}
% \begin{axis}[
%   ybar stacked,
%   width=12cm,
%   height=8cm,
%   title={Impact of Hidden Layer Size on Accuracy},
%   ylabel={Accuracy},
%   bar width=20pt,
%   %ymin=0.2,
%   ymax=1,
%   yticklabel={\pgfmathparse{\tick*100}\pgfmathprintnumber{\pgfmathresult}\,\%},
%   ytick style={draw=none},
%   ymajorgrids = true,
%   symbolic x coords={op+type+size/hidden-10, op+type+size/hidden-25, op+type+size/hidden-50,
%                      op+type+size/hidden-100, op+type+size/hidden-250, op+type+size/hidden-500},
%   % enlarge x limits=0.25,
%   xtick=data,
%   xtick style={draw=none},
%   xticklabels={\hiddenT, \hiddenTF, \hiddenF, \hiddenH, \hiddenTHF, \hiddenFH},
%   x tick label style={rotate=45},
%   reverse legend,
%   legend style={legend pos = north west},
% ]
% \addplot[draw=black, fill=blue1] table[x=tool, y=top-1] {\HiddenBench};
% \addplot[draw=black, fill=blue2] table[x=tool, y expr=\thisrow{top-2} - \thisrow{top-1}] {\HiddenBench};
% \addplot[draw=black, fill=blue3] table[x=tool, y expr=\thisrow{top-3} - \thisrow{top-2}] {\HiddenBench};
% % \addplot[draw=black, fill=blue1] table[x=tool, y=top-1] {\HiddenBench};
% % \addplot[draw=black, fill=blue2] table[x=tool, y=top-2] {\HiddenBench};
% % \addplot[draw=black, fill=blue3] table[x=tool, y=top-3] {\HiddenBench};
% \legend{Top 1, Top 2, Top 3}
% \end{axis}
% \end{tikzpicture}
% \caption{reuslts!}
% \label{fig:results}
% \end{figure}

\makeatletter
\newcommand\resetstackedplots{
\makeatletter
\pgfplots@stacked@isfirstplottrue
\makeatother
\addplot [forget plot,draw=none] coordinates{
  (baseline,0) (spacer1,0)
  (ocaml,0) (mycroft,0) (sherrloc,0) (spacer2,0)
  (op+type+size/linear,0)
  (op+type+size/decision-tree,0)
  (op+type+size/hidden-10,0)
  (op+type+size/hidden-250,0)
};
}
\makeatother

\begin{figure}[ht]
\centering
\begin{tikzpicture}
\begin{axis}[
  ybar stacked,
  width=14cm,
  height=8cm,
  title={Comparison of Type-Error Localization Techniques},
  ylabel={Accuracy},
  bar width=0.5cm,
  ymin=0.2,
  ymax=1,
  yticklabel={\pgfmathparse{\tick*100}\pgfmathprintnumber{\pgfmathresult}\,\%},
  ytick style={draw=none},
  ymajorgrids = true,
  symbolic x coords={baseline, spacer1, ocaml, mycroft, sherrloc, spacer2,
                     op+type+size/linear,
                     op+type+size/decision-tree,
                     op+type+size/hidden-10,
                     op+type+size/hidden-250},
  % enlarge x limits=0.5,
  xtick=data,
  xtick style={draw=none},
  xticklabels={\baseline, \ocaml, \mycroft, \sherrloc,
               \linear, \dectree, \hiddenT, \hiddenTHF},
  x tick label style={rotate=45},
  reverse legend,
  transpose legend,
  legend style={legend pos = north west, legend columns=4},
]
\addplot[draw=black, fill=blue1, bar shift=-.25cm] table[x=tool, y=top-1] {\SpringBench};
\addplot[draw=black, fill=blue2, bar shift=-.25cm] table[x=tool, y expr=\thisrow{top-2} - \thisrow{top-1}] {\SpringBench};
\addplot[draw=black, fill=blue3, bar shift=-.25cm] table[x=tool, y expr=\thisrow{top-3} - \thisrow{top-2}] {\SpringBench};
\addlegendentry{Top 1}
\addlegendentry{Top 2}
\addlegendentry{Top 3}
\addlegendimage{empty legend}
\addlegendentry{FA15}

\resetstackedplots

\addplot[draw=black, fill=blue1, bar shift=.25cm] table[x=tool, y=top-1] {\FallBench};
\addplot[draw=black, fill=blue2, bar shift=.25cm] table[x=tool, y expr=\thisrow{top-2} - \thisrow{top-1}] {\FallBench};
\addplot[draw=black, fill=blue3, bar shift=.25cm] table[x=tool, y expr=\thisrow{top-3} - \thisrow{top-2}] {\FallBench};
\addlegendentry{Top 1}
\addlegendentry{Top 2}
\addlegendentry{Top 3}
\addlegendentry{SP14}
\addlegendimage{empty legend}
%\legend{Top 1, Top 2, Top 3}
\end{axis}
\end{tikzpicture}
\caption{reuslts!}
\label{fig:results}
\end{figure}






\subsection{Qualitative Evaluation}
\label{sec:qualitative}

Next, we present a \emph{qualitative} evaluation that compares the
predictions made by our classifiers with those of \sherrloc.
%
In particular, we demonstrate, with a series of example programs from
our student dataset, how our classifiers are able to use past student
mistakes to make more accurate predictions of future fixes.
%
For each example, we provide
%
(1) the code,
%
(2) \sherrloc's prediction (in \textbf{bold}), and
%
(3) the prediction of our Decision Tree (\underline{underlined}).
%
We choose the Decision Tree classifier for this section as its model
is more easily interpreted than the MLP.
%
We will also attempt, for each program, to explain \emph{why} the
Decision Tree made its prediction, by analyzing the paths induced
by the programs.

\paragraph{Extracting the Digits of an Integer}
% data/sp14/1655.ml
Our first program is a simple recursive function |digitsOfInt| that
extracts the digits of an |int|.
%
\begin{ecode}
  let rec digitsOfInt n =
    if n <= 0 then
      []
    else
      [n mod 10] @ __[ ___=digitsOfInt (n / 10)=___ ]__
\end{ecode}
%
Unfortunately, the student has decided to wrap the recursive call to
|digitsOfInt| with a list literal, even though |digitsOfInt| already
returns an |int list|.
%
Thus, the list literal is inferred to have type |int list list|, which
is incompatible with the |int list| on the left of the |@| (list append)
operator.
%
Both \sherrloc and the \ocaml compiler blame the recursive call for
returning a |int list| rather than |int|, but the recursive call is
actually correct!
%
As our Decision Tree correctly points out, the fault lies with the list
literal \emph{surrounding} the recursive call, remove it and the type
error disappears.
%
\ES{TODO: WHY?}

\paragraph{Padding a list}
% data/sp14/0306.ml
Our next program, |padZero|, is given two |int list|s as input, and must
left-pad the shorter one with enough zeros that the two output lists
have equal length.

The student first defines a helper function |clone|, which takes an
input element |x| and an |int| |n|, and returns a list containing |n|
|x|s.
%
\begin{ecode}
  let rec clone x n =
    if n <= 0 then
      []
    else
      x :: clone x (n - 1)
\end{ecode}
%
Then she defines |padZero| with a simple branch that determines which
list is shorter, followed by a |clone| with the appropriate number of
zeros.
%
\lstset{firstnumber=last}
\begin{ecode}
  let padZero l1 l2 =
    let n = List.length l1 - List.length l2 in
    if n < 0 then
      (clone 0 ((-1) * n) @ l2, l2)
    else
      (l1, _=clone 0 n=___ :: l2__)
\end{ecode}
\lstset{firstnumber=1}
%
Alas, our student has accidentally used the |::| operator rather than
the |@| operator in the |else| branch.
%
\sherrloc and \ocaml correctly determine that she cannot cons the
|int list| returned by |clone| onto |l2|, which is another |int list|,
but they decide to \emph{blame} the call to |clone|, while our Decision
Tree correctly blames the |::| operator.
%
\ES{TODO: WHY? (might be too similar to previous?)}


% ES: decision tree gets this one wrong, may want to find something else
% \begin{ecode}
%   let rec sepConcat sep sl =
%     match sl with
%     | [] -> ""
%     | h::t ->
%         let f a x = a ^ (sep ^ x) in
%         List.fold_left f h t

%   let stringOfList f l = sepConcat "; " __[ "["; ___=List.map f l=___; "]" ]__
% \end{ecode}

\paragraph{Computing the Fixed Point of a Function}
% data/sp14/0941.ml
Finally, our students must write a |fixpoint| function that computes the
fixed point of a given function |f|, starting from an initial value |b|.
%
As a hint we first have them write a |wwhile| function that performs the
functional equivalent of a \texttt{while}-loop, repeatedly passing a function's
output back in until it receives a (boolean) signal to stop.
%
\begin{ecode}
  let rec wwhile (f, b) =
    match f b with
    | (x, false) -> x
    | (x, true)  -> wwhile (f, x)
\end{ecode}
\lstset{firstnumber=last}
%
The |fixpoint| function can then be written as a clever instantiation of
the arguments to |wwhile|.
%
\begin{ecode}
  let fixpoint (f, b) =
    let g = __let bb = f b in ___=(bb, (bb = b))=_ in
    wwhile (g, b)
\end{ecode}
\lstset{firstnumber=1}
%
Sadly, our student has forgotten that |g| should itself be a function.
%
As a result, she passes a pair of a \emph{pair} and a starting value to
|wwhile|, rather than a pair of a \emph{function} and a starting value.
%
\sherrloc deduces that the call to |wwhile| is likely correct (\ocaml
actually blames the use of |g|), but identifies the construction of the
pair inside |g| as the most likely culprit, while our Decision Tree
correctly identifies the \emph{definition} of |g| as the source of the
error.
\ES{TODO: WHY?}


\subsubsection{Failed Predictions}
\label{sec:failed-predictions}
Of course, our classifiers are sometimes wrong, we focus next on programs
where our classifier makes an incorrect prediction.

\paragraph{Constructing a List of Duplicates}
% data/sp14/0148.ml
Consider the following implementation of the |clone| function from
before.
%
\begin{ecode}
  let rec clone x n =
    let loop acc n =
      if n <= 0 then
        acc
      else
        clone ==([==_=x=_==] @ acc)== (n - 1) in
    loop [] n
\end{ecode}
% \ES{TODO: our 2nd prediction matches \sherrloc and \ocaml (occurs check), correct fix is to replace recursive call to clone with loop}
%
The student has defined a helper function |loop| with an accumulator
|acc|, likely meant to call itself tail-recursively.
%
Unfortunately, she has called the top-level function |clone| rather than
|loop| in the |else| branch, this induces a cyclic constraint |'a = 'a list|
for the |x| argument to |clone|.

Our classifier incorrectly predicts that the use of |x| in the recursive
call is the most likely source of the error.
\ES{TOOD: WHY??}
%
Our second prediction coincides with \sherrloc (and \ocaml), blaming the
the first argument to |clone|.
%
This is also incorrect, but may be more helpful than our first
prediction --- if our student decides that she has certainly provided
the correct \emph{argument}, an alternative explanation is that
perhaps she has called the wrong \emph{function}.
%
Our final prediction is the entire body of |clone|, which is hardly
helpful.

\paragraph{Currying Considered Harmful?}
% data/sp14/0887.ml
Our next example is another ill-fated attempt at |clone|.
%
\begin{ecode}
  let rec clone x n =
    let rec loop x n acc =
      if n < 0 then
        acc
      else
        ==loop== __(x, (n - 1), (x :: acc))__ in
    loop (x, n, [])
\end{ecode}
The issue here is that \ocaml functions are \emph{curried} by default
--- \ie they take their arguments one at a time --- but our student has
called the inner |loop| with all three arguments in a tuple.
%
Many experienced functional programmers would choose to keep |loop|
curried and rewrite the calls, however our student decides instead to
\emph{uncurry} |loop|, making it take a tuple of arguments.
%
\sherrloc blames the recursive call to |loop| while our classifier
blames the tuple of arguments --- a reasonable suggestion, but alas not
the answer the student was looking for.
\ES{WHY??}
%

\paragraph{Composing Functions}
% data/sp14/2269.ml
Next, let us consider the |pipe| function that composes a list of
functions, \ie given a list of functions |[f;g;h]|, |pipe| produces the
function |fun x -> h (g (f x))|.
%
\begin{ecode}
  let pipe fs =
    let f a x y = ==y== __(a y)__ in
    let base x = x in
    List.fold_left f base fs
\end{ecode}
%
The error in our student's code is that she has applied |y| rather than
|x| to the result of |a y|.
%
\sherrloc correctly blames the first occurrence of |y|, while our
classifier (incorrectly) blames the application |a y| (\ocaml blames
the occurrence of |base| on line 4).
%
\ES{TODO: WHY?}


%%% Local Variables:
%%% mode: latex
%%% TeX-master: "main"
%%% End:


%%% Local Variables:
%%% mode: latex
%%% TeX-master: "main"
%%% End:

\mysection{Related Work}
\label{sec:related-work}
\label{sec:type-error-diagnosis}

In this section we describe two relevant aspects of related work:
%
programming languages approaches to diagnosing type errors, and
%
software engineering approaches to fault localization.
%
% applying machine learning techniques to problems in programming languages.
%
% \mysubsection{Type Error Diagnosis}
%% Languages with static type systems and type inference produce type
%% errors that novices often perceive as difficult to
%% interpret~\citep{Wand1986-nw}.
%% %
%% For example, approaches based on Hindley-Milner or other constraint
%% systems typically have the issue that when the constraint $\alpha=\beta$
%% is violated, the system cannot know whether the user should intend to
%% fix $\alpha$ or $\beta$ and must thus report the discrepancy in a
%% generic manner.
%% %
%% As a result, a number of approaches have been proposed to
%% more precisely localize the type error,
%% give clearer error messages, or
%% fix the error automatically.
%
% The technique most related to our work is \ES{TODO}.  It has been shown
% to be quite good at \ES{TODO}. However, we choose to focus on \ES{TODO},
% and our approach contains \ES{TODO}, which is not present in previous
% work.

\mypara{Localizing Type Errors}
It is well-known that the original Damas-Milner
algorithm $\mathcal{W}$ produces errors far
from their source, that novices percieve as
difficult to interpret~\citep{Wand1986-nw}.
%
%% The type checker collects typing constraints as it traverses the
%% program, and it crucially assumes that if a constraint can be safely
%% added to the set of assumptions (\ie no type error has been found yet),
%% then the constraint can be \emph{assumed} to be correct.
%
The type checker reports an error the moment
it finds a constraint that contradicts one
of the assumptions, blaming the new inconsistent
constraint, and thus it is extremely sensitive
to the order in which it traverses the source
program (the infamous ``left-to-right''
bias~\citep{McAdam1998-ub}).
%
Several alternative traversal have been proposed,
\eg top-down rather than bottom-up~\citep{Lee1998-ys},
or a \emph{symmetric} traversal that checks
sub-expressions independently and only reports an
error when two inconsistent sets of constraints are
merged~\citep{McAdam1998-ub,Yang1999-yr}.
%
%\mypara{Slicing}
Type error \emph{slicing}~\citep{Haack2003-vc,Tip2001-qp,Rahli2010-ps}
overcomes the constraint-order bias by extracting a
complete and minimal subset
of terms that contribute to the error, \ie all of the
terms that are required for it to manifest and no more.
%
Slicing typically requires rewriting the type checker with a
specialized constraint language and solver, though
\citet{Schilling2011-yf} shows how to turn any type checker into a
slicer by treating it as a black-box.
%
While slicing techniques guarantee enough information to diagnose the
error, they can fall into the trap of providing \emph{too much}
information, producing a slice that is not much smaller than
the input. %original. % input program.

\mypara{Finding Likely Errors}
Thus, recent work has focused on finding the \emph{most likely} source
of a type error.
%
\citet{Zhang2014-lv} use Bayesian reasoning to search the constraint
graph for constraints that participate in many unsatisfiable paths and
relatively few satisfiable paths, based on the intuition that the
program should be mostly correct.
%
\citet{Pavlinovic2014-mr} translate the localization problem into a
MaxSMT problem, asking an off-the-shelf solver to find the smallest
set of constraints that can be removed such that the resulting system is
satisfiable.
%
\citet{Loncaric2016-uk} improve the scalability of
\citeauthor{Pavlinovic2014-mr} by reusing the existing type checker as
a theory solver in the Nelson-Oppen~\citeyear{Nelson1979-td}
style, and thus require only a MaxSAT solver.
%
All three of these techniques support \emph{weighted} constraints to
incorporate knowledge about the frequency of different errors,
but only \citeauthor{Pavlinovic2014-mr} use the weights, setting them to
the size of the term that induced the constraint.
%
In contrast, our classifiers learn a set of heuristics for predicting
the source of type errors by observing a set of ill-typed programs and
their subsequent fixes, in a sense using \emph{only} the weights and no
constraint solver.
%
It may be profitable to combine both approaches, \ie learn a set of good
weights for one of the above techniques from our training data.

\mypara{Explaining Type Errors}
In this paper we have focused solely on the task of \emph{localizing} a
type error, but a good error report should also \emph{explain} the
error.
%
\citet{Wand1986-nw}, \citet{Beaven1993-hb}, and \citet{Duggan1996-by}
attempt to explain type errors by collecting the chain of inferences
made by the type checker %--- essentially the typing derivation ---
and presenting them to the user.
%
% Such explanations can be lengthy, as an attempt to compress the
% explanation \citet{Yang2000-kz} presents a visualization of the
% inference process.
%
\citet{Gast2004-zd} produces a slice enhanced by arrows
showing the dataflow from sources with different types to a
shared sink, borrowing the insight of dataflows-as-explanations from
\textsc{MrSpidey}~\citep{Flanagan1996-bu}.
%
\citet{Hage2006-hc} catalog a set of heuristics for
improving the quality of error messages by examining errors made by
novices.
%
\citet{Heeren2003-db}, \citet{Christiansen2014-qc}, and
\citet{Serrano2016-oo} extend the ability to customize error messages to
library authors, enabling \emph{domain-specific} errors.
%
%\mypara{Interactive Explanations}
Such \emph{static} explanations of type errors run the risk of
overwhelming the user with too much information, it may be preferable to
treat type error diagnosis as an \emph{interactive} debugging session.
%
\citet{Bernstein1995-yj} extend the type inference procedure to handle
\emph{open} expressions (\ie with unbound variables), allowing users to
interactively query the type checker for the types of sub-expressions.
%
\citet{Chitil2001-td} proposes \emph{algorithmic debugging} of type
errors, presenting the user with a sequence of yes-or-no questions about
the inferred types of sub-expressions that guide the user to a specific
explanation for the error.
%
\citet{Seidel2016-ul} explain type errors by searching for inputs that
expose the \emph{run-time} error that the type system prevented, and
present users with an interactive visualization of the erroneous
computation.

%% RJ-CUT-ME \mypara{Programmatic Explanations}
%% RJ-CUT-ME %
%% RJ-CUT-ME The best explanation of a type error, however, might be given by an
%% RJ-CUT-ME expert, \eg the compiler or library author.
%% RJ-CUT-ME %
%% RJ-CUT-ME \citet{Hage2006-hc} catalog a set of heuristics for
%% RJ-CUT-ME improving the quality of error messages by examining errors made by
%% RJ-CUT-ME novices.
%% RJ-CUT-ME %
%% RJ-CUT-ME \citet{Heeren2003-db}, \citet{Christiansen2014-qc}, and
%% RJ-CUT-ME \citet{Serrano2016-oo} extend the ability to customize error messages to
%% RJ-CUT-ME library authors, enabling \emph{domain-specific} errors.
%
% The 8.0 release of the
% Glasgow Haskell Compiler\footnote{\url{https://ghc.haskell.org/trac/ghc/wiki/Proposal/CustomTypeErrors}}
% incorporates these ideas, allowing library authors to supply
% custom errors when type-class resolution or type-family reduction fail,
% but not for ordinary unification failures.


\mypara{Fixing Type Errors}
Some techniques go beyond explaining or locating a type error,
and actually attempt to \emph{fix} the error automatically.
%
\citet{Lerner2007-dt} searches for fixes by enumerating a
set of local mutations to the program and querying the type checker to
see if the error remains.
%
\citet{Chen2014-gd} use a notion of \emph{variation-based} typing to
track choices made by the type checker and enumerate potential
% type (and expression)
changes that would fix the error.
%
They also extend the algorithmic debugging technique of
\citeauthor{Chitil2001-td} by allowing the user to enter the expected
type of specific sub-expressions and suggesting fixes based on these
desired types \citeyear{Chen2014-vm}.
%
Our classifiers do not attempt to suggest fixes to type errors, but it
may be possible to do so by training a classifier to predict the
syntactic class of each expression in the \emph{fixed} program --- we
believe this is an exciting direction for future work.

\mypara{Fault Localization}
%
Given a defect, \emph{fault localization} is the task of identifying
``suspicious'' program elements (\eg lines, statements) that are likely
implicated in the defect %(\ie that should be changed to fix the defect)
%
--- thus, type error localization can be viewed as an instance of fault
localization.
%
The best-known fault localization technique is likely Tarantula, which
uses a simple mathematical formula based on measured information from
dynamic normal and buggy runs~\citep{Jones2002-us}.
%
Other similar approaches, including those of \citet{Chen2002-qz} and
\citet{Abreu2006-fn,Abreu2007-mu} consider alternate features of
information or refined formulae and generally obtain more precise
results; see \citet{Wong2009-pd} for a survey.
%
While some researchers have approached such fault localization with an
eye toward optimality (\eg \citet{Yoo2013-rw} determine optimal
coefficients), in general such fault localization approaches are limited
by their reliance on either running tests or including relevant
features.
%
For example, Tarantula-based techniques require a normal and a buggy run
of the program.
%
By contrast, we consider incomplete programs with type errors that may
not be executed in any standard sense.
%
Similarly, the features available influence the classes of defects that
can be localized.
%
For example, a fault localization scheme based purely on control flow features
will have difficulty with cross-site scripting or SQL code injection
attacks, which follow the same control flow path on normal and buggy
runs (differing only in the user-supplied data).
%
Our feature set is comprised entirely of syntactic and typing features,
a natural choice for modeling type errors, but it would likely not
generalize to other types of defects.


% \subsection{Machine Learning for Programming Languages}
% \label{sec:ml-pl}



%%% Local Variables:
%%% mode: latex
%%% TeX-master: "main"
%%% End:



%% Acknowledgments
\begin{acks}                            %% acks environment is optional
                                        %% contents suppressed with 'anonymous'
  %% Commands \grantsponsor{<sponsorID>}{<name>}{<url>} and
  %% \grantnum[<url>]{<sponsorID>}{<number>} should be used to
  %% acknowledge financial support and will be used by metadata
  %% extraction tools.
  This material is based upon work supported by the
  \grantsponsor{GS100000001}{National Science
    Foundation}{http://dx.doi.org/10.13039/100000001} under Grant
  No.~\grantnum{GS100000001}{nnnnnnn} and Grant
  No.~\grantnum{GS100000001}{mmmmmmm}.  Any opinions, findings, and
  conclusions or recommendations expressed in this material are those
  of the author and do not necessarily reflect the views of the
  National Science Foundation.
\end{acks}


%% Bibliography
\bibliography{main}


%% Appendix
% \appendix
% \section{Appendix}

% Text of appendix \ldots

\end{document}
