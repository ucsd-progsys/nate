\subsection{Blame Utility}
\label{sec:user-study}

%\ES{TODO: explain that we selected programs where nate got the right answer and sherrloc did NOT}

% Finally, to test the explanatory power of our blame assigments, we
% ran a user study (IRB \#2014009900) at the University of Virginia (UVA).
% %
% We included three problems in an exam in the Spring 2017 session of UVA's
% undergraduate Programming Languages course (CS 4501).
% %
% We presented the 31 students in the course with ill-typed \ocaml\
% programs and asked them to
% %
% (1) \emph{explain} the type error, and
% %
% (2) \emph{fix} the type error.
% %
% For each problem the student was given the ill-typed program and
% either \sherrloc or \toolname's blame assignment, with no error message.

We assigned three problems to the students: the |padZero| and
|mulByDigit| programs from \autoref{sec:qualitative}, as well as
the following |sepConcat| program
%
\begin{ecode}
  let rec sepConcat sep sl =
    match sl with
    | [] -> ""
    | h::t ->
        let f a x = a ^ (sep ^ x) in
        let base = (*@\hlTree{[]}@*) in
        (*@\hlSherrloc{List.fold\_left f base sl}@*)
\end{ecode}
%
where the student has erroneously returned the empty list, rather than
the empty string, in the base case of the fold.
% which triggers an error on line 7.
%
% For each problem the students were given the ill-typed program and
% either \ocaml's error or \toolname's jump-compressed trace;
%
The full user study is available in \autoref{sec:user-study-exams}.
%
Due to the nature of an in-class exam not every student answered every
question, but we always received at least 12 (out of a possible 15 or
16) responses for each problem-tool pair.
%
A further complication for this study is that this session of the course
was taught in \textsc{Reason}\footnote{\url{https://reasonml.github.io}},
a dialect of \ocaml with a more C-like syntax, and thus for the study
we transcribed the programs to \textsc{Reason} syntax.

We then instructed three annotators (one of whom is an author, the others
are graduate students at UCSD) to classify the answers as
correct or incorrect.
%
We performed an inter-rater reliability (IRR) analysis to determine the
degree to which the annotators consistently graded the exams.
%
As we had more than two annotators assigning nominal (``correct'' or
``incorrect'') ratings we used Fleiss' kappa~\cite{Fleiss1971-du} to
measure IRR.\@
%
Fleiss' kappa is measured on a scale from $1$, indicating total
agreement, to $-1$, indicating total disagreement, with $0$ indicating
random agreement.

Finally, we used a one-sided Mann-Whitney $U$ test~\cite{Mann1947-fd} to
determine the significance of our results.
%
The null hypothesis was that the responses from students given
\toolname's blame were drawn from the same distribution as those
given \sherrloc's, \ie \toolname had no effect.
%
Since we used a one-sided test, the alternative to the null hypothesis
is that \toolname had a \emph{positive} effect on the responses.
%
We reject the null hypothesis in favor of the alternative if the test
produces a significance level $p < 0.05$, a standard threshold for
determining statistical significance.

\mypara{Threats to Validity}
%
Measuring understanding is difficult, and comes with its own
set of threats. % to validity.

\paragraph{Construct.}
%
We used the correctness of the student's explanation of, and fix for,
the type error as a proxy for her understanding, but it is possible
that other metrics would produce different results.
%
A further threat arises from our decision to use \textsc{Reason} syntax
rather than \ocaml.
%
\textsc{Reason} and \ocaml differ only in syntax, the type system is the
same; however, the difference in syntax may affect students'
understanding of the programs.
%
For example, \textsc{Reason} uses the notation |[h, ...t]| for the list
``cons'' constructor, in contrast to \ocaml's |h::t|.
%
It is quite possible that \textsc{Reason}'s syntax could help students
remember that |h| is a single element while |t| is a list.

\paragraph{Internal.}
%
We assigned students randomly to two groups. The first was given
\sherrloc's blame assignment for |sepConcat| and |mulByDigit|, and
\toolname's blame for |padZero|; the second was given the opposite
assignment. This ensured that each student was given \sherrloc and
\toolname problems. Students without sufficient knowledge of
\textsc{Reason} could affect the results, as could the time-constrained
nature of an exam. Thus, we excluded any answers left blank
from our analysis.

\paragraph{External.}
%
Our experiment used students in the process of learning \textsc{Reason},
and thus may not generalize to all developers. The three programs were
chosen manually, via a random selection and filtering of the programs
from the \SPRING dataset, where \toolname's top prediction was correct
but \sherrloc's was incorrect. A different selection of programs may
lead to different results.

\paragraph{Conclusion.}
%
We collected exams from 31 students, though due to the nature of the
study not every student completed every problem.
%
The number of complete submissions was always at least 12 out of
a maximum of 15 or 16 per program-tool pair.
%
While our results are not statistically significant,
collecting more responses per test pair was not possible as it
would require having students answer the same problem twice (once with
\sherrloc and once with \toolname).

\begin{figure}[t]
\centering
%% EXPLAIN
\begin{tikzpicture}
\begin{axis}[
  ybar,
  % width=6cm,
  height=6cm,
  title={\large Explanation},
  ylabel={\% Correct},
  bar width=0.5cm,
  ymin=0,
  ymax=1,
  yticklabel={\pgfmathparse{\tick*100}\pgfmathprintnumber{\pgfmathresult}\,\%},
  ytick style={draw=none},
  ymajorgrids = true,
  enlarge x limits=0.25,
  symbolic x coords={{sepConcat}, {padZero}, {mulByDigit}},
  xtick=data,
  xtick style={draw=none},
  xticklabel style={align=center},
  xticklabels={
    {\texttt{sepConcat}\\($p=0.48$)},
    {\texttt{padZero}\\($p=0.097$)},
    {\texttt{mulByDigit}\\($p=0.083$)}
  },
  % x tick label style={rotate=45, anchor=north east},
  %x tick label style={font=\small},
  y tick label style={font=\small},
  reverse legend,
  %transpose legend,
  legend style={legend pos = south west, legend columns=2, font=\footnotesize},
]

% ES: NOTE: ORDER OF PLOTS/LEGEND ENTRIES MATTERS

\addplot[draw=black, fill=green2, bar shift=.25cm, error bars/.cd, y dir=both, y explicit] coordinates {
  (sepConcat, 0.8125)  +- (0, 0.0976)
  (padZero, 1.00)      +- (0, 0)
  (mulByDigit, 0.8125) +- (0, 0.0976)
};
\addlegendentry{\toolname}

\addplot[draw=black, fill=blue2, bar shift=-.25cm, error bars/.cd, y dir=both, y explicit] coordinates {
  (sepConcat, 0.80)    +- (0, 0.1033)
  (padZero, 0.875)     +- (0, 0.0827)
  (mulByDigit, 0.5714) +- (0, 0.1323)
};
\addlegendentry{\sherrloc}
\end{axis}
\end{tikzpicture}
%% FIX
\begin{tikzpicture}
\begin{axis}[
  ybar,
  % width=6cm,
  height=6cm,
  title={\large Fix},
  ylabel={\% Correct},
  bar width=0.5cm,
  ymin=0,
  ymax=1,
  yticklabel={\pgfmathparse{\tick*100}\pgfmathprintnumber{\pgfmathresult}\,\%},
  ytick style={draw=none},
  ymajorgrids = true,
  enlarge x limits=0.25,
  symbolic x coords={{sepConcat}, {padZero}, {mulByDigit}},
  xtick=data,
  xtick style={draw=none},
  xticklabel style={align=center},
  xticklabels={
    {\texttt{sepConcat}\\($p=0.57$)},
    {\texttt{padZero}\\($p=0.33$)},
    {\texttt{mulByDigit}\\($p=0.31$)}
  },
  % x tick label style={rotate=45, anchor=north east},
  %x tick label style={font=\small},
  y tick label style={font=\small},
  reverse legend,
  %transpose legend,
  legend style={legend pos = south west, legend columns=2, font=\footnotesize},
]

% ES: NOTE: ORDER OF PLOTS/LEGEND ENTRIES MATTERS

\addplot[draw=black, fill=green2, bar shift=.25cm, error bars/.cd, y dir=both, y explicit] coordinates {
  (sepConcat, 0.8125)  +- (0, 0.0976)
  (padZero, 0.9286)    +- (0, 0.0688)
  (mulByDigit, 0.7143) +- (0, 0.1207)
};
\addlegendentry{\toolname}

\addplot[draw=black, fill=blue2, bar shift=-.25cm, error bars/.cd, y dir=both, y explicit] coordinates {
  (sepConcat, 0.83)    +- (0, 0.1076)
  (padZero, 0.875)     +- (0, 0.0827)
  (mulByDigit, 0.6154) +- (0, 0.1349)
};
\addlegendentry{\sherrloc}
\end{axis}
\end{tikzpicture}
\caption[A classification of students' explanations and fixes for type
  errors, given either \sherrloc or \toolname's blame assignment.]
  {A classification of students' explanations and fixes for type
  errors, given either \sherrloc or \toolname's blame assignment.
  %
  The students given \toolname's location generally scored
  better than those given \sherrloc's.
  %
  We report the result of a one-sided Mann-Whitney $U$ test for
  statistical significance in parentheses.}
\label{fig:results-user-study}
\end{figure}


% \begin{figure}[t]
% \centering
% % \centerline{
% % \begin{minipage}{1.2\textwidth}
% % \centering
% \includegraphics[width=0.49\linewidth]{user-study-reason.png}
% \includegraphics[width=0.49\linewidth]{user-study-fix.png}
% % \end{minipage}
% % }
% % \vspace{3ex}
% \caption[A classification of students' explanations and fixes for type
%   errors, given either \sherrloc or \toolname's blame assignment.]
%   {A classification of students' explanations and fixes for type
%   errors, given either \sherrloc or \toolname's blame assignment.
%   %
%   The students given \toolname's location generally scored
%   better than those given \sherrloc's.
%   %
%   We report the result of a one-sided Mann-Whitney $U$ test for
%   statistical significance in parentheses.}
% \label{fig:results-user-study}
% \end{figure}

\mypara{Results}
%
The measured kappa values were $\kappa = 0.68$ for the explanations and
$\kappa = 0.77$ for the fixes; while there is no formal notion for what
consititutes strong agreement~\cite{Krippendorff2012-wd}, kappa values
above $0.60$ are often called ``substantial''
agreement~\cite{Landis1977-ey}.
%
Figure~\ref{fig:results-user-study} summarizes a single annotator's
results, which show that students given \toolname's blame assignment
were generally more likely to correctly explain and fix the type error
than those given \sherrloc's.
%
While there was no discernible difference between \toolname and
\sherrloc for |sepConcat|, \toolname responses for |padZero| and
|mulByDigit| were marked correct 5--30\% more often than the \sherrloc
responses.
%
While the results appear to show a trend in favor of \toolname,
they do not rise to the level of statistical significance.
% The results appear to show a trend in favor of \toolname;
% %
% however, none rise to the level of statistical significance.
%
